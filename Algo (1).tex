\documentclass[a4paper, 12pt]{article}
\usepackage{algorithm}
\usepackage[utf8]{inputenc}
\usepackage[T1]{fontenc}
\usepackage[french]{babel}
\usepackage{amsmath}
\usepackage{amssymb}
\usepackage{hyperref}
\usepackage{tabularx}
\usepackage{array}
\usepackage{algpseudocode}

\pagestyle{headings}

\title{ASSEMBLEUR : Projet}
\author{DELAFORGE Jean, PELOUX Louis}
\date{\today}

\begin{document}

\maketitle

\newpage

\tableofcontents

\newpage




\section{Programme Jules Cesar}

\subsection{Pseudo-code}

\begin{center}
  \begin{tabular}{|l|}
    \hline
    \multicolumn{1}{|c|}{ Algorithme Jules César 2} \\
    \hline
    Entrée : 2 arguments, le premier un entier et le second une chaîne de caractères  \\
    \hline
    Sortie : Affichage de la chaîne modifiée \\
    \hline
  \end{tabular}
\end{center}

\begin{algorithm}
\caption{Décalage des caractères d'une chaîne}
\begin{algorithmic}[1]
\State \textbf{début} :
\State $i \leftarrow 0$
\State $clé \leftarrow \text{depiler}(P)$
\State $clé \leftarrow \text{atoi}(clé)$
\State $chaîne \leftarrow \text{depiler}(P)$
\While{$chaîne \neq$ vide}
    \State $c \leftarrow \text{chaîne}[i]$
    \While{$c > 122$}
        \State $c \leftarrow c - 26$
    \EndWhile
    \State $\text{chaîne}[i] \leftarrow c + clé$
    \State $i \leftarrow i + 1$
\EndWhile
\State Afficher la chaîne modifiée
\State \textbf{fin}
\end{algorithmic}
\end{algorithm}


\subsection{Fonctionnement du code}

L'algorithme prend en parametre 2 arguments, le premier correspond a la clé, celle-ci doit être passée en argument d'une fonction atoi pour pouvoir avoir sa valeur en entier.
Le second argument est une chaîne de caractères.
\\
\\
Premièrement, on verifie qu'il y est bien le bon nombre d'arguments, c'est-à-dire 2 arguments. Si ce n'est pas le cas, on affiche le message d'erreur "Il faut 2 arguments !!" dans stdout et on finit le programme.
\\
On initialise un registre un 0 qui aura pour but d'être l'indice de la chaîne de caractères.
\\
\\
Ensuite, on recupère la chaîne de caractères contenue dans la pile et on la stocke dans un registre. On parcours le registre jusqu'à arriver au bout de la chaîne. On recupère le caractère à l'indice, on lui ajoute a son code ascii la clé et on réécrit la valeur dans la chaîne.
\\
On recommence cette boucle tant que la chaîne argv2 soit vide.
Enfin, on affiche la chaîne de caractères modifiée.
\\

\newpage

\section{Programme Vigenère}

\subsection{Pseudo-code}

\begin{center}
  \begin{tabular}{|l|}
    \hline
    \multicolumn{1}{|c|}{ Algorithme Vigenere 2} \\
    \hline
    Entrée : Deux chaînes de caractères, une qui fait office de clé, et l'autre à décoder  \\
    \hline
    Sortie : Chaîne de caractères modifiée \\
    \hline
  \end{tabular}
\end{center}

\begin{algorithm}
\caption{Algorithme Vigenère}
\begin{algorithmic}[1]
\State \textbf{début :}
\State $i \gets 0$
\State $j \gets 0$
\State $k \gets 0$
\State $clé \gets \text{depiler}(P)$
\State $chaîne \gets \text{depiler}(P)$
\State $chaîne3 \gets '\backslash 0'$ \Comment{Initialisation de la chaîne codée}
\While{$chaîne$ n'est pas vide}
    \State $c \gets \text{chaîne}[i]$
    \If{$clef[j]$ est vide}
        \State $j \gets 0$ \Comment{Boucle sur la clé}
    \EndIf
    \If{$A \leq c \leq Z$}
        \State $c \gets c + 32$ \Comment{Convertir en minuscule}
    \EndIf
    \If{$a \leq c \leq z$}
        \State $h \gets clé[j]$
        \State $h \gets h - 97$ \Comment{Décalage selon la clé}
        \State $c \gets c + h$
        \While{$c > 122$}
            \State $c \gets c - 26$ \Comment{Reste dans la plage des minuscules}
        \EndWhile
        \State $chaîne3[k] \gets c$
        \State $j \gets j + 1$
    \Else
        \State Appeler \texttt{sous\_vigenere}
        \State $chaîne3[i] \gets '.'$
    \EndIf
    \State $i \gets i + 1$
    \State $k \gets k + 1$
\EndWhile
\State Afficher $chaîne3$
\State \textbf{fin}
\end{algorithmic}
\end{algorithm}


\newpage


\subsection{Fonctionnement du code}

L'algorithme de Vigenère commence par dépiler la clé et le message à chiffrer. Il initialise trois registres : i, j, et k, qui seront utilisés pour parcourir respectivement le message, la clé, et la chaîne codée.
\\
\\
À chaque itération, l'algorithme récupère un caractère du message (chaine[i]) et, si le caractère est valide (lettre de l'alphabet), le décalage est effectué en fonction de la clé. Si le caractère est une lettre majuscule, il est d'abord converti en minuscule.
\\
\\
Le décalage est effectué en ajoutant la valeur de la clé à la valeur ASCII du caractère, et si cette somme dépasse la lettre 'z', une boucle permet de revenir dans la plage des lettres minuscules.
\\
\\
Si un caractère n'est pas valide (pas une lettre), l'algorithme passe à la fonction \texttt{sous\_vigenere} pour vérifier et passer au caractère suivant. Enfin, la chaîne codée est affichée à la fin du processus.

\newpage

\subsection{Pseudo-code Sous Vigenère}
\begin{algorithm}
\caption{Vérification de la validité du caractère}
\begin{algorithmic}[1]
\State \textbf{début :}
\State $fin\_chaine \gets 1$
\While{la chaîne n'est pas terminée}
    \State $i \gets i + 1$ \Comment{(i = indice chaîne)}
    \State $c \gets \text{chaine}[i]$
    \If{$c = 0$}
        \State $fin\_chaine \gets 0$
    \EndIf
    \If{$A \leq c \leq Z$}
        \State \textbf{return}
    \ElsIf{$a \leq c \leq z$}
        \State \textbf{return}
    \EndIf
\EndWhile
\State \textbf{fin}
\end{algorithmic}
\end{algorithm}

\subsection{Fonctionnement du code sous\_vigenere}

La fonction \texttt{sous\_vigenere} est utilisée pour vérifier si un caractère dans la chaîne à chiffrer (\texttt{chaine}) est une lettre valide (majuscule ou minuscule).

Elle commence par initialiser une variable \texttt{fin\_chaine} à 1, ce qui indique que la chaîne est valide au début. Ensuite, elle boucle sur chaque caractère de la chaîne à partir de l'indice \(i\).

\begin{itemize}
    \item Si le caractère est nul (fin de chaîne), la fonction définit \texttt{fin\_chaine} à 0 et termine la vérification.
    \item Si le caractère est une lettre majuscule (de A à Z), la fonction \texttt{return} pour indiquer qu'il s'agit d'un caractère valide.
    \item Si le caractère est une lettre minuscule (de a à z), la fonction \texttt{return}.
    \item Si le caractère n'est ni une lettre majuscule ni une lettre minuscule, la fonction continue la boucle pour vérifier le prochain caractère.
\end{itemize}

Si la boucle se termine sans trouver de caractère valide, la fonction quitte sans retourner de résultat explicite. Sa logique sert à gérer la progression à travers les caractères de la chaîne jusqu'à ce qu'un caractère valide soit trouvé.

\end{document}