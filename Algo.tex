\documentclass[a4paper, 12pt]{article}
\usepackage[utf8]{inputenc}
\usepackage[T1]{fontenc}
\usepackage[french]{babel}
\usepackage{amsmath}
\usepackage{amssymb}
\usepackage{hyperref}
\usepackage{tabularx}
\usepackage{array}

\pagestyle{headings}

\title{ASSEMBLEUR : Projet}
\author{DELAFORGE Jean, PELOUX Louis}
\date{\today}

\begin{document}

\maketitle

\newpage

\tableofcontents

\newpage




\section{Programme Jules Cesar}

\subsection{Pseudo-code}

\begin{center}
  \begin{tabular}{|l|}
    \hline
    \multicolumn{1}{|c|}{ Algorithme Jules Cesar 2} \\
    \hline
    Entrée : 2 arguments, le premier un entier et le second une chaine de caractere  \\
    \hline
    Sortie : Affichage de la chaine modifier \\
    \hline
  \end{tabular}
\end{center}


\begin{center}
  \begin{tabularx}{\textwidth}{|X|}
    \hline
    début : \\
    \hspace{1em}$i \leftarrow 0$ ; \\
    \hspace{1em}$clef \leftarrow depiler(P)$ ; \\
    \hspace{1em}atoi($clef$); \\ 
    \hspace{1em}$chaine \leftarrow depiler(P)$ ; \\
    \hspace{1em}tant que $chaine$ n'est pas vide faire \\
    \hspace{2em}$c \leftarrow chaine$[i]; \\
    \hspace{2em}Tant que $c$ > 122  Alors \\
    \hspace{3em}$c \leftarrow c$ - 26;\\
    \hspace{2em}Fin tant que \\
    \hspace{2em}$chaine[i] \leftarrow c + clef;$ \\
    \hspace{2em}$i \leftarrow i + 1;$ \\
    \hspace{1em}Fin tant que \\
    \hspace{1em}On affiche la chaine modifier \\
    fin \\
    \hline
  \end{tabularx}
\end{center}


\subsection{Fonctionnement du code}

L'algorythme prend en parametre 2 arguments, le premier correspond a la clef, celle ci doit être passer en argument d'une fonction atoi pour pouvoir avoir la valeur en entier.
Le second argument est une chaine de caracteres.
\\
\\
Premièrement on verifie qu'il y est bien le bon nombre d'argument, c'est a dire 2 arguments. Si ce n'est pas le cas, on affiche le message d'erreur "Il faut 2 arguments !!" dans stdout et on finit le programme.
\\
On initialise un registre un 0 qui aura pour but d'être l'indice de la chaine de caracteres.
\\
\\
Ensuite on recupere la chaine de caracteres contenue dans la pile et on le stocke dans un registre. On parcours le registre jusqu'a arriver au bout de la chaine. On reccupère le caracteres a l'indice, on lui ajoute a son code ascii la clef et on recrit la valeur dans la chaine.
\\
On recommence cette boucle tant que la chaîne argv2 soit vide.
Enfin, on affiche la chaine de caracteres modifier.
\\

\newpage

\section{Programme du Code de Vigenere}


\subsection{Pseudo-code}

\begin{center}
  \begin{tabularx}{\textwidth}{|X|}
    \hline
    \multicolumn{1}{|c|}{ Algorithme de Vigenere } \\
    \hline
    Entrée : Deux chaines de caracteres, une qui fait office de clef, et l'autre a decoder  \\
    \hline
    Sortie : Chaine de caracteres modifier  \\
    \hline
  \end{tabularx}
\end{center}


\begin{center}
  \begin{tabularx}{\textwidth}{|X|}
    \hline
    début : \\
    \hspace{1em}$chaine1 \leftarrow depiler(P)$ ;\\
    \hspace{1em}$chaine2 \leftarrow depiler(P)$ ;\\
    \hspace{1em}$chaine3 \leftarrow '\textbackslash 0' ;\\
    \hspace{1em}$i \leftarrow 0$ \\
    \hspace{1em}$j \leftarrow 0$ \\
    \hspace{1em}$k \leftarrow 0$ \\ 
    \hspace{1em}tant que chaine2 n'est pas vide alors\\
    \hspace{2em}$c \leftarrow chaine2[i]$ \\
    \hspace{2em}Si chaine1[j] vide alors\\
    \hspace{3em}$ j \leftarrow 0$ \\
    \hspace{2em}Fin si \\
    \hspace{2em}$ h \leftarrow chaine1[j]$ \\
    \hspace{2em}Si $c$ n'est ni une majuscule ou minuscule alors \\
    \hspace{3em}appeler $sous\textunderscore vigenere;\\
    \hspace{2em}fin si\\
    \hspace{2em}$h \leftarrow h - 97$;\\
    \hspace{2em}$c \leftarrow c + h ;\\
    \hspace{2em}Tant que c > 122 Alors \\
    \hspace{3em}c \leftarrow c - 26$ \\
    \hspace{2em}Fin Tant que\\
    \hspace{2em}$chaine3[k] \leftarrow c$ ;\\
    \hspace{2em}i \leftarrow i + 1;\\
    \hspace{2em}j \leftarrow i + 1;\\
    \hspace{2em}k \leftarrow i + 1;\\
    \hspace{1em}Fin Tant que\\  
    fin \\
    \hline
  \end{tabularx}
\end{center}

\subsection{Pseudo-code Sous Vigenere}

\begin{center}
  \begin{tabularx}{\textwidth}{|X|}
    \hline
    \multicolumn{1}{|c|}{ Algorithme de Sous Vigenere } \\
    \hline
    Entrée : Une chaine de caractères et un index pour avancer dans la chaine \\
    \hline
    Sortie : Indication si la chaine est terminée ou avancée jusqu'au prochain caractère valide \\
    \hline
  \end{tabularx}
\end{center}

\begin{center}
  \begin{tabularx}{\textwidth}{|X|}
    \hline
    début : \\
    \hspace{1em}$fin\_chaine \leftarrow 1$ ;\\
    \hspace{1em}Tant que la chaîne n'est pas terminée alors \\
    \hspace{2em}$i \leftarrow i + 1$ ; (i = indice chaîne)\\
    \hspace{2em}$c \leftarrow \text{chaine2}[i]$ ; \\
    \hspace{2em}Si $c = 0$ alors \\
    \hspace{3em}$fin\_chaine \leftarrow 0$ ; \\
    \hspace{2em}Fin si \\
    \hspace{2em}Si $c \geq A$ et $c \leq Z$ alors \\
    \hspace{3em}Retourner oui ; \\
    \hspace{2em}Sinon \\
    \hspace{3em}Si $c \geq a$ et $c \leq z$ alors \\
    \hspace{4em}Retourner oui ; \\
    \hspace{3em}Sinon : \\
    \hspace{4em}Revenir au début de la boucle \\
    \hspace{3em}Fin si\\
    \hspace{2em}Fin si\\
    \hspace{1em}Fin tant que \\

    \hline
  \end{tabularx}
\end{center}

\end{document}
